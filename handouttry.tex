\documentclass[
  man,
  floatsintext,
  longtable,
  nolmodern,
  notxfonts,
  notimes,
  colorlinks=true,linkcolor=blue,citecolor=blue,urlcolor=blue]{apa7}

\usepackage{amsmath}
\usepackage{amssymb}



\usepackage[bidi=default]{babel}
\babelprovide[main,import]{american}


% get rid of language-specific shorthands (see #6817):
\let\LanguageShortHands\languageshorthands
\def\languageshorthands#1{}

\RequirePackage{longtable}
\RequirePackage{threeparttablex}

\makeatletter
\renewcommand{\paragraph}{\@startsection{paragraph}{4}{\parindent}%
	{0\baselineskip \@plus 0.2ex \@minus 0.2ex}%
	{-.5em}%
	{\normalfont\normalsize\bfseries\typesectitle}}

\renewcommand{\subparagraph}[1]{\@startsection{subparagraph}{5}{0.5em}%
	{0\baselineskip \@plus 0.2ex \@minus 0.2ex}%
	{-\z@\relax}%
	{\normalfont\normalsize\bfseries\itshape\hspace{\parindent}{#1}\textit{\addperi}}{\relax}}
\makeatother




\usepackage{longtable, booktabs, multirow, multicol, colortbl, hhline, caption, array, float, xpatch}
\usepackage{subcaption}
\renewcommand\thesubfigure{\Alph{subfigure}}
\setcounter{topnumber}{2}
\setcounter{bottomnumber}{2}
\setcounter{totalnumber}{4}
\renewcommand{\topfraction}{0.85}
\renewcommand{\bottomfraction}{0.85}
\renewcommand{\textfraction}{0.15}
\renewcommand{\floatpagefraction}{0.7}

\usepackage{tcolorbox}
\tcbuselibrary{listings,theorems, breakable, skins}
\usepackage{fontawesome5}

\definecolor{quarto-callout-color}{HTML}{909090}
\definecolor{quarto-callout-note-color}{HTML}{0758E5}
\definecolor{quarto-callout-important-color}{HTML}{CC1914}
\definecolor{quarto-callout-warning-color}{HTML}{EB9113}
\definecolor{quarto-callout-tip-color}{HTML}{00A047}
\definecolor{quarto-callout-caution-color}{HTML}{FC5300}
\definecolor{quarto-callout-color-frame}{HTML}{ACACAC}
\definecolor{quarto-callout-note-color-frame}{HTML}{4582EC}
\definecolor{quarto-callout-important-color-frame}{HTML}{D9534F}
\definecolor{quarto-callout-warning-color-frame}{HTML}{F0AD4E}
\definecolor{quarto-callout-tip-color-frame}{HTML}{02B875}
\definecolor{quarto-callout-caution-color-frame}{HTML}{FD7E14}

%\newlength\Oldarrayrulewidth
%\newlength\Oldtabcolsep


\usepackage{hyperref}



\usepackage{color}
\usepackage{fancyvrb}
\newcommand{\VerbBar}{|}
\newcommand{\VERB}{\Verb[commandchars=\\\{\}]}
\DefineVerbatimEnvironment{Highlighting}{Verbatim}{commandchars=\\\{\}}
% Add ',fontsize=\small' for more characters per line
\usepackage{framed}
\definecolor{shadecolor}{RGB}{241,243,245}
\newenvironment{Shaded}{\begin{snugshade}}{\end{snugshade}}
\newcommand{\AlertTok}[1]{\textcolor[rgb]{0.68,0.00,0.00}{#1}}
\newcommand{\AnnotationTok}[1]{\textcolor[rgb]{0.37,0.37,0.37}{#1}}
\newcommand{\AttributeTok}[1]{\textcolor[rgb]{0.40,0.45,0.13}{#1}}
\newcommand{\BaseNTok}[1]{\textcolor[rgb]{0.68,0.00,0.00}{#1}}
\newcommand{\BuiltInTok}[1]{\textcolor[rgb]{0.00,0.23,0.31}{#1}}
\newcommand{\CharTok}[1]{\textcolor[rgb]{0.13,0.47,0.30}{#1}}
\newcommand{\CommentTok}[1]{\textcolor[rgb]{0.37,0.37,0.37}{#1}}
\newcommand{\CommentVarTok}[1]{\textcolor[rgb]{0.37,0.37,0.37}{\textit{#1}}}
\newcommand{\ConstantTok}[1]{\textcolor[rgb]{0.56,0.35,0.01}{#1}}
\newcommand{\ControlFlowTok}[1]{\textcolor[rgb]{0.00,0.23,0.31}{\textbf{#1}}}
\newcommand{\DataTypeTok}[1]{\textcolor[rgb]{0.68,0.00,0.00}{#1}}
\newcommand{\DecValTok}[1]{\textcolor[rgb]{0.68,0.00,0.00}{#1}}
\newcommand{\DocumentationTok}[1]{\textcolor[rgb]{0.37,0.37,0.37}{\textit{#1}}}
\newcommand{\ErrorTok}[1]{\textcolor[rgb]{0.68,0.00,0.00}{#1}}
\newcommand{\ExtensionTok}[1]{\textcolor[rgb]{0.00,0.23,0.31}{#1}}
\newcommand{\FloatTok}[1]{\textcolor[rgb]{0.68,0.00,0.00}{#1}}
\newcommand{\FunctionTok}[1]{\textcolor[rgb]{0.28,0.35,0.67}{#1}}
\newcommand{\ImportTok}[1]{\textcolor[rgb]{0.00,0.46,0.62}{#1}}
\newcommand{\InformationTok}[1]{\textcolor[rgb]{0.37,0.37,0.37}{#1}}
\newcommand{\KeywordTok}[1]{\textcolor[rgb]{0.00,0.23,0.31}{\textbf{#1}}}
\newcommand{\NormalTok}[1]{\textcolor[rgb]{0.00,0.23,0.31}{#1}}
\newcommand{\OperatorTok}[1]{\textcolor[rgb]{0.37,0.37,0.37}{#1}}
\newcommand{\OtherTok}[1]{\textcolor[rgb]{0.00,0.23,0.31}{#1}}
\newcommand{\PreprocessorTok}[1]{\textcolor[rgb]{0.68,0.00,0.00}{#1}}
\newcommand{\RegionMarkerTok}[1]{\textcolor[rgb]{0.00,0.23,0.31}{#1}}
\newcommand{\SpecialCharTok}[1]{\textcolor[rgb]{0.37,0.37,0.37}{#1}}
\newcommand{\SpecialStringTok}[1]{\textcolor[rgb]{0.13,0.47,0.30}{#1}}
\newcommand{\StringTok}[1]{\textcolor[rgb]{0.13,0.47,0.30}{#1}}
\newcommand{\VariableTok}[1]{\textcolor[rgb]{0.07,0.07,0.07}{#1}}
\newcommand{\VerbatimStringTok}[1]{\textcolor[rgb]{0.13,0.47,0.30}{#1}}
\newcommand{\WarningTok}[1]{\textcolor[rgb]{0.37,0.37,0.37}{\textit{#1}}}

\providecommand{\tightlist}{%
  \setlength{\itemsep}{0pt}\setlength{\parskip}{0pt}}
\usepackage{longtable,booktabs,array}
\usepackage{calc} % for calculating minipage widths
% Correct order of tables after \paragraph or \subparagraph
\usepackage{etoolbox}
\makeatletter
\patchcmd\longtable{\par}{\if@noskipsec\mbox{}\fi\par}{}{}
\makeatother
% Allow footnotes in longtable head/foot
\IfFileExists{footnotehyper.sty}{\usepackage{footnotehyper}}{\usepackage{footnote}}
\makesavenoteenv{longtable}

\usepackage{graphicx}
\makeatletter
\newsavebox\pandoc@box
\newcommand*\pandocbounded[1]{% scales image to fit in text height/width
  \sbox\pandoc@box{#1}%
  \Gscale@div\@tempa{\textheight}{\dimexpr\ht\pandoc@box+\dp\pandoc@box\relax}%
  \Gscale@div\@tempb{\linewidth}{\wd\pandoc@box}%
  \ifdim\@tempb\p@<\@tempa\p@\let\@tempa\@tempb\fi% select the smaller of both
  \ifdim\@tempa\p@<\p@\scalebox{\@tempa}{\usebox\pandoc@box}%
  \else\usebox{\pandoc@box}%
  \fi%
}
% Set default figure placement to htbp
\def\fps@figure{htbp}
\makeatother


% definitions for citeproc citations
\NewDocumentCommand\citeproctext{}{}
\NewDocumentCommand\citeproc{mm}{%
  \begingroup\def\citeproctext{#2}\cite{#1}\endgroup}
\makeatletter
 % allow citations to break across lines
 \let\@cite@ofmt\@firstofone
 % avoid brackets around text for \cite:
 \def\@biblabel#1{}
 \def\@cite#1#2{{#1\if@tempswa , #2\fi}}
\makeatother
\newlength{\cslhangindent}
\setlength{\cslhangindent}{1.5em}
\newlength{\csllabelwidth}
\setlength{\csllabelwidth}{3em}
\newenvironment{CSLReferences}[2] % #1 hanging-indent, #2 entry-spacing
 {\begin{list}{}{%
  \setlength{\itemindent}{0pt}
  \setlength{\leftmargin}{0pt}
  \setlength{\parsep}{0pt}
  % turn on hanging indent if param 1 is 1
  \ifodd #1
   \setlength{\leftmargin}{\cslhangindent}
   \setlength{\itemindent}{-1\cslhangindent}
  \fi
  % set entry spacing
  \setlength{\itemsep}{#2\baselineskip}}}
 {\end{list}}
\usepackage{calc}
\newcommand{\CSLBlock}[1]{\hfill\break\parbox[t]{\linewidth}{\strut\ignorespaces#1\strut}}
\newcommand{\CSLLeftMargin}[1]{\parbox[t]{\csllabelwidth}{\strut#1\strut}}
\newcommand{\CSLRightInline}[1]{\parbox[t]{\linewidth - \csllabelwidth}{\strut#1\strut}}
\newcommand{\CSLIndent}[1]{\hspace{\cslhangindent}#1}





\usepackage{newtx}

\defaultfontfeatures{Scale=MatchLowercase}
\defaultfontfeatures[\rmfamily]{Ligatures=TeX,Scale=1}





\title{The Importance of Summary Statistics and Techniques for Creating
Them in R}


\shorttitle{Summary Statistics and Techniques in R}


\usepackage{etoolbox}






\author{Jelin George (Matriculation Number: 400826617)}



\affiliation{
{Hochschule Fresenius - University of Applied Science}}




\leftheader{Number: 400826617)}



\abstract{This document presents a concise overview of summary
statistics and their importance in R. Summary statistics - such as mean,
median, standard deviation, and frequency counts - capture the key
features of a dataset, enabling quick exploration and interpretation. R
provides powerful functions and visualization tools to efficiently
compute and present these statistics, making them essential for
simplifying data, identifying patterns, and supporting informed analysis
and decision-making. Practical examples and code are provided to
demonstrate these concepts in action. The document also discusses the
limitations of summary statistics and its application. }

\keywords{summary statistics, R programming, data analysis, descriptive
statistics, data visualization, exploratory data analysis, limitations}

\authornote{ 
\par{ }
\par{   The authors have no conflicts of interest to disclose.    }
\par{Correspondence concerning this article should be addressed to Jelin
George (Matriculation Number:
400826617), Email: \href{mailto:george.jelin@stud-hs.fresenius.de}{george.jelin@stud-hs.fresenius.de}}
}

\makeatletter
\let\endoldlt\endlongtable
\def\endlongtable{
\hline
\endoldlt
}
\makeatother

\urlstyle{same}



\usepackage{booktabs}
\usepackage{longtable}
\usepackage{array}
\usepackage{multirow}
\usepackage{wrapfig}
\usepackage{float}
\usepackage{colortbl}
\usepackage{pdflscape}
\usepackage{tabu}
\usepackage{threeparttable}
\usepackage{threeparttablex}
\usepackage[normalem]{ulem}
\usepackage{makecell}
\usepackage{xcolor}
\makeatletter
\@ifpackageloaded{caption}{}{\usepackage{caption}}
\AtBeginDocument{%
\ifdefined\contentsname
  \renewcommand*\contentsname{Table of contents}
\else
  \newcommand\contentsname{Table of contents}
\fi
\ifdefined\listfigurename
  \renewcommand*\listfigurename{List of Figures}
\else
  \newcommand\listfigurename{List of Figures}
\fi
\ifdefined\listtablename
  \renewcommand*\listtablename{List of Tables}
\else
  \newcommand\listtablename{List of Tables}
\fi
\ifdefined\figurename
  \renewcommand*\figurename{Figure}
\else
  \newcommand\figurename{Figure}
\fi
\ifdefined\tablename
  \renewcommand*\tablename{Table}
\else
  \newcommand\tablename{Table}
\fi
}
\@ifpackageloaded{float}{}{\usepackage{float}}
\floatstyle{ruled}
\@ifundefined{c@chapter}{\newfloat{codelisting}{h}{lop}}{\newfloat{codelisting}{h}{lop}[chapter]}
\floatname{codelisting}{Listing}
\newcommand*\listoflistings{\listof{codelisting}{List of Listings}}
\makeatother
\makeatletter
\makeatother
\makeatletter
\@ifpackageloaded{caption}{}{\usepackage{caption}}
\@ifpackageloaded{subcaption}{}{\usepackage{subcaption}}
\makeatother

% From https://tex.stackexchange.com/a/645996/211326
%%% apa7 doesn't want to add appendix section titles in the toc
%%% let's make it do it
\makeatletter
\xpatchcmd{\appendix}
  {\par}
  {\addcontentsline{toc}{section}{\@currentlabelname}\par}
  {}{}
\makeatother

%% Disable longtable counter
%% https://tex.stackexchange.com/a/248395/211326

\usepackage{etoolbox}

\makeatletter
\patchcmd{\LT@caption}
  {\bgroup}
  {\bgroup\global\LTpatch@captiontrue}
  {}{}
\patchcmd{\longtable}
  {\par}
  {\par\global\LTpatch@captionfalse}
  {}{}
\apptocmd{\endlongtable}
  {\ifLTpatch@caption\else\addtocounter{table}{-1}\fi}
  {}{}
\newif\ifLTpatch@caption
\makeatother

\begin{document}

\maketitle


\setcounter{secnumdepth}{-\maxdimen} % remove section numbering

\setlength\LTleft{0pt}


Summary statistics are concise numerical measures that capture the
essential characteristics of a dataset. They serve as foundational tools
in data analysis, providing concise descriptions of large datasets. They
help analysts and researchers understand the central tendencies,
variability, and overall distribution of data, making complex datasets
interpretable and actionable. Without summary statistics, raw data would
be overwhelming and difficult to interpret, making it challenging to
draw meaningful conclusions or communicate findings effectively.

In R, summary statistics are foundational for data analysis, enabling
users to efficiently condense complex data into interpretable values
like the mean, median, mode, standard deviation, and quantiles. R offers
a rich ecosystem of functions and packages, each with unique features.
Base R provides basic summaries, while dplyr allows flexible, tidy group
summaries, skimr and summarytools create detailed, readable overviews,
often with visual elements. Packages like psych, Hmisc, and pastecs
offer more advanced descriptive statistics. For publication-ready
tables, gtsummary and table1 are ideal. Additional tools such as
janitor, rstatix, and doBy support quick tabulation and custom
summaries, giving users a range of options for data analysis.

As the first and often most critical step in any analytical workflow,
summary statistics in R empower analysts and researchers to understand,
compare, and communicate data-driven insights with clarity and
precision.

Summary statistics can be typically divided into:

\begin{enumerate}
\def\labelenumi{\arabic{enumi}.}
\item
  \textbf{Descriptive statistics}: Summarize the main features of a
  dataset (e.g., mean, median, mode). \emph{This will be our focus
  here.}
\item
  \textbf{Inferential statistics}: Make predictions or inferences about
  a population based on a sample (not the focus here).
\end{enumerate}

I would like to highlight a book, \emph{Making sense of statistics: A
conceptual overview}, (\citeproc{ref-oh2023making}{Oh \& Pyrczak, 2023})
which offers a clear and accessible introduction to key statistical
concepts for beginners. The book focuses on building conceptual
understanding of both descriptive and inferential statistics, using
simple explanations, practical examples, and step-by-step guidance. It
is designed to help in applying statistics to research and interpreting
data effectively.

For a deeper exploration of R packages used for summarizing data, I
encourage you to visit the source by
(\citeproc{ref-medcalf2018favourite}{Medcalf, 2018})

Additonally, watch this
\href{https://www.youtube.com/watch?v=yoPGwvUzjgQ}{\textbf{tutorial
video}} on descriptive statistics in R to get you started.

\section{Key Measures in Summary
Statistics}\label{key-measures-in-summary-statistics}

Summary statistics simplify complex datasets into a few key numbers,
making it easier to understand and communicate the main characteristics
of the data.

The primary measures include central tendency (mean, median, and mode),
which describe where most values fall, and measures of dispersion, which
capture the spread of the data. Measures of shape and distribution, such
as skewness and kurtosis, provide insight into the overall pattern and
extremities of the data.

Visualization tools like histograms and boxplots help illustrate these
patterns and highlight outliers or skewness. Handling missing data is
also important, as different types of missingness (MCAR, MAR, MNAR) can
bias results if not addressed through omission or imputation.

Frequency tables and cross-tabulations organize and display how often
values or categories occur, making it easier to spot patterns and
summarize large datasets. Finally, summarizing entire data frames with
these statistics offers a comprehensive overview of each variable,
revealing trends and important features within the dataset.

Furthermore, read
\href{https://modernstatisticswithr.com/index.html}{Modern Statistics
with R} to understand essential tools and techniques in contemporary
statistical data analysis, using the R programming language. The book
features numerous examples and over 200 exercises with worked solutions.
The online version is freely available and regularly updated, with
downloadable datasets for hands-on learning

The YouTube videos referenced here may assist in further understanding
the code chunks presented above
(\citeproc{ref-walker2023gtsummary}{Walker, 2023})
(\citeproc{ref-dre2024gentle}{Videos, 2024})
(\citeproc{ref-Schork2021}{Schork, 2021})

\section{Practical Application}\label{practical-application}

Try this exercise to understand how to read data and apply summary
statistics functions using the \textbf{Star Wars} dataset.

Before we get started, we must install essential packages that might be
needed later.

\begin{Shaded}
\begin{Highlighting}[]
\ControlFlowTok{if}\NormalTok{ (}\SpecialCharTok{!}\FunctionTok{require}\NormalTok{(pacman)) }\FunctionTok{install.packages}\NormalTok{(}\StringTok{"pacman"}\NormalTok{)}
\NormalTok{pacman}\SpecialCharTok{::}\FunctionTok{p\_load}\NormalTok{(tidyverse)}
\end{Highlighting}
\end{Shaded}

Load the Star Wars dataset available in the dylyr package. Read more on
dylyr package here (\citeproc{ref-dplyr2023}{Wickham et al., 2023})

\begin{Shaded}
\begin{Highlighting}[]
\FunctionTok{library}\NormalTok{(dplyr)}
\FunctionTok{data}\NormalTok{(starwars)}
\end{Highlighting}
\end{Shaded}

To begin our analysis, we will display the first 10 rows of the starwars
dataset. This provides a quick overview of the data structure and its
key variables before we proceed with summary statistics.

\begin{Shaded}
\begin{Highlighting}[]
\NormalTok{starwars\_tbl }\OtherTok{\textless{}{-}}\NormalTok{ starwars }\SpecialCharTok{\%\textgreater{}\%}
  \FunctionTok{slice\_head}\NormalTok{(}\AttributeTok{n =} \DecValTok{10}\NormalTok{)}

\FunctionTok{kable}\NormalTok{(starwars\_tbl, }\AttributeTok{format =} \StringTok{"latex"}\NormalTok{, }\AttributeTok{booktabs =} \ConstantTok{TRUE}\NormalTok{, }\AttributeTok{caption =} \StringTok{"Table 1. Star Wars Data"}\NormalTok{) }\SpecialCharTok{\%\textgreater{}\%}
  \FunctionTok{kable\_styling}\NormalTok{(}\AttributeTok{latex\_options =} \StringTok{"striped"}\NormalTok{, }\AttributeTok{full\_width =} \ConstantTok{FALSE}\NormalTok{)}
\end{Highlighting}
\end{Shaded}

\begin{table}
\centering
\caption{\label{tab:unnamed-chunk-5}Table 1. Star Wars Data}
\centering
\begin{tabular}[t]{lrrlllrlllllll}
\toprule
name & height & mass & hair\_color & skin\_color & eye\_color & birth\_year & sex & gender & homeworld & species & films & vehicles & starships\\
\midrule
\cellcolor{gray!10}{Luke Skywalker} & \cellcolor{gray!10}{172} & \cellcolor{gray!10}{77} & \cellcolor{gray!10}{blond} & \cellcolor{gray!10}{fair} & \cellcolor{gray!10}{blue} & \cellcolor{gray!10}{19.0} & \cellcolor{gray!10}{male} & \cellcolor{gray!10}{masculine} & \cellcolor{gray!10}{Tatooine} & \cellcolor{gray!10}{Human} & \cellcolor{gray!10}{A New Hope             , The Empire Strikes Back, Return of the Jedi     , Revenge of the Sith    , The Force Awakens} & \cellcolor{gray!10}{Snowspeeder          , Imperial Speeder Bike} & \cellcolor{gray!10}{X-wing          , Imperial shuttle}\\
C-3PO & 167 & 75 & NA & gold & yellow & 112.0 & none & masculine & Tatooine & Droid & A New Hope             , The Empire Strikes Back, Return of the Jedi     , The Phantom Menace     , Attack of the Clones   , Revenge of the Sith &  & \\
\cellcolor{gray!10}{R2-D2} & \cellcolor{gray!10}{96} & \cellcolor{gray!10}{32} & \cellcolor{gray!10}{NA} & \cellcolor{gray!10}{white, blue} & \cellcolor{gray!10}{red} & \cellcolor{gray!10}{33.0} & \cellcolor{gray!10}{none} & \cellcolor{gray!10}{masculine} & \cellcolor{gray!10}{Naboo} & \cellcolor{gray!10}{Droid} & \cellcolor{gray!10}{A New Hope             , The Empire Strikes Back, Return of the Jedi     , The Phantom Menace     , Attack of the Clones   , Revenge of the Sith    , The Force Awakens} & \cellcolor{gray!10}{} & \cellcolor{gray!10}{}\\
Darth Vader & 202 & 136 & none & white & yellow & 41.9 & male & masculine & Tatooine & Human & A New Hope             , The Empire Strikes Back, Return of the Jedi     , Revenge of the Sith &  & TIE Advanced x1\\
\cellcolor{gray!10}{Leia Organa} & \cellcolor{gray!10}{150} & \cellcolor{gray!10}{49} & \cellcolor{gray!10}{brown} & \cellcolor{gray!10}{light} & \cellcolor{gray!10}{brown} & \cellcolor{gray!10}{19.0} & \cellcolor{gray!10}{female} & \cellcolor{gray!10}{feminine} & \cellcolor{gray!10}{Alderaan} & \cellcolor{gray!10}{Human} & \cellcolor{gray!10}{A New Hope             , The Empire Strikes Back, Return of the Jedi     , Revenge of the Sith    , The Force Awakens} & \cellcolor{gray!10}{Imperial Speeder Bike} & \cellcolor{gray!10}{}\\
\addlinespace
Owen Lars & 178 & 120 & brown, grey & light & blue & 52.0 & male & masculine & Tatooine & Human & A New Hope          , Attack of the Clones, Revenge of the Sith &  & \\
\cellcolor{gray!10}{Beru Whitesun Lars} & \cellcolor{gray!10}{165} & \cellcolor{gray!10}{75} & \cellcolor{gray!10}{brown} & \cellcolor{gray!10}{light} & \cellcolor{gray!10}{blue} & \cellcolor{gray!10}{47.0} & \cellcolor{gray!10}{female} & \cellcolor{gray!10}{feminine} & \cellcolor{gray!10}{Tatooine} & \cellcolor{gray!10}{Human} & \cellcolor{gray!10}{A New Hope          , Attack of the Clones, Revenge of the Sith} & \cellcolor{gray!10}{} & \cellcolor{gray!10}{}\\
R5-D4 & 97 & 32 & NA & white, red & red & NA & none & masculine & Tatooine & Droid & A New Hope &  & \\
\cellcolor{gray!10}{Biggs Darklighter} & \cellcolor{gray!10}{183} & \cellcolor{gray!10}{84} & \cellcolor{gray!10}{black} & \cellcolor{gray!10}{light} & \cellcolor{gray!10}{brown} & \cellcolor{gray!10}{24.0} & \cellcolor{gray!10}{male} & \cellcolor{gray!10}{masculine} & \cellcolor{gray!10}{Tatooine} & \cellcolor{gray!10}{Human} & \cellcolor{gray!10}{A New Hope} & \cellcolor{gray!10}{} & \cellcolor{gray!10}{X-wing}\\
Obi-Wan Kenobi & 182 & 77 & auburn, white & fair & blue-gray & 57.0 & male & masculine & Stewjon & Human & A New Hope             , The Empire Strikes Back, Return of the Jedi     , The Phantom Menace     , Attack of the Clones   , Revenge of the Sith & Tribubble bongo & Jedi starfighter        , Trade Federation cruiser, Naboo star skiff        , Jedi Interceptor        , Belbullab-22 starfighter\\
\bottomrule
\end{tabular}
\end{table}

\begin{Shaded}
\begin{Highlighting}[]
\CommentTok{\# Mean height}
\NormalTok{starwars }\SpecialCharTok{\%\textgreater{}\%} 
  \FunctionTok{summarise}\NormalTok{(}\AttributeTok{mean\_height =} \FunctionTok{mean}\NormalTok{(height, }\AttributeTok{na.rm =} \ConstantTok{TRUE}\NormalTok{))}
\end{Highlighting}
\end{Shaded}

\begin{verbatim}
# A tibble: 1 x 1
  mean_height
        <dbl>
1        175.
\end{verbatim}

\begin{Shaded}
\begin{Highlighting}[]
\CommentTok{\# Median height}
\NormalTok{starwars }\SpecialCharTok{\%\textgreater{}\%} 
  \FunctionTok{summarise}\NormalTok{(}\AttributeTok{median\_height =} \FunctionTok{median}\NormalTok{(height, }\AttributeTok{na.rm =} \ConstantTok{TRUE}\NormalTok{))}
\end{Highlighting}
\end{Shaded}

\begin{verbatim}
# A tibble: 1 x 1
  median_height
          <int>
1           180
\end{verbatim}

\begin{Shaded}
\begin{Highlighting}[]
\CommentTok{\# Mode height}

\NormalTok{starwars }\SpecialCharTok{\%\textgreater{}\%}
  \FunctionTok{filter}\NormalTok{(}\SpecialCharTok{!}\FunctionTok{is.na}\NormalTok{(height)) }\SpecialCharTok{\%\textgreater{}\%}
  \FunctionTok{count}\NormalTok{(height, }\AttributeTok{sort =} \ConstantTok{TRUE}\NormalTok{) }\SpecialCharTok{\%\textgreater{}\%}
  \FunctionTok{slice\_max}\NormalTok{(}\AttributeTok{n =} \DecValTok{1}\NormalTok{, }\AttributeTok{order\_by =}\NormalTok{ n) }\SpecialCharTok{\%\textgreater{}\%}
  \FunctionTok{select}\NormalTok{(}\AttributeTok{mode\_height =}\NormalTok{ height)}
\end{Highlighting}
\end{Shaded}

\begin{verbatim}
# A tibble: 1 x 1
  mode_height
        <int>
1         183
\end{verbatim}

Now, let's apply other summary functions.

\begin{Shaded}
\begin{Highlighting}[]
\CommentTok{\# Select relevant variables}
\NormalTok{starwars\_selected }\OtherTok{\textless{}{-}}\NormalTok{ starwars }\SpecialCharTok{\%\textgreater{}\%}
  \FunctionTok{select}\NormalTok{(height, mass, gender, birth\_year, species)}

\CommentTok{\# Tidy summary for numeric variables}
\NormalTok{starwars\_selected }\SpecialCharTok{\%\textgreater{}\%}
  \FunctionTok{summarise}\NormalTok{(}
    \AttributeTok{mean\_height =} \FunctionTok{mean}\NormalTok{(height, }\AttributeTok{na.rm =} \ConstantTok{TRUE}\NormalTok{),}
    \AttributeTok{sd\_height =} \FunctionTok{sd}\NormalTok{(height, }\AttributeTok{na.rm =} \ConstantTok{TRUE}\NormalTok{),}
    \AttributeTok{mean\_mass =} \FunctionTok{mean}\NormalTok{(mass, }\AttributeTok{na.rm =} \ConstantTok{TRUE}\NormalTok{),}
    \AttributeTok{sd\_mass =} \FunctionTok{sd}\NormalTok{(mass, }\AttributeTok{na.rm =} \ConstantTok{TRUE}\NormalTok{)}
\NormalTok{  )}
\end{Highlighting}
\end{Shaded}

\begin{verbatim}
# A tibble: 1 x 4
  mean_height sd_height mean_mass sd_mass
        <dbl>     <dbl>     <dbl>   <dbl>
1        175.      34.8      97.3    169.
\end{verbatim}

\begin{Shaded}
\begin{Highlighting}[]
\CommentTok{\# Frequency table for gender (tidyverse style)}
\NormalTok{starwars\_selected }\SpecialCharTok{\%\textgreater{}\%}
  \FunctionTok{count}\NormalTok{(gender, }\AttributeTok{name =} \StringTok{"frequency"}\NormalTok{)}
\end{Highlighting}
\end{Shaded}

\begin{verbatim}
# A tibble: 3 x 2
  gender    frequency
  <chr>         <int>
1 feminine         17
2 masculine        66
3 <NA>              4
\end{verbatim}

\begin{Shaded}
\begin{Highlighting}[]
\CommentTok{\# Proportion table for gender (tidyverse style)}
\NormalTok{starwars\_selected }\SpecialCharTok{\%\textgreater{}\%}
  \FunctionTok{count}\NormalTok{(gender, }\AttributeTok{name =} \StringTok{"frequency"}\NormalTok{) }\SpecialCharTok{\%\textgreater{}\%}
  \FunctionTok{mutate}\NormalTok{(}\AttributeTok{proportion =}\NormalTok{ frequency }\SpecialCharTok{/} \FunctionTok{sum}\NormalTok{(frequency))}
\end{Highlighting}
\end{Shaded}

\begin{verbatim}
# A tibble: 3 x 3
  gender    frequency proportion
  <chr>         <int>      <dbl>
1 feminine         17     0.195 
2 masculine        66     0.759 
3 <NA>              4     0.0460
\end{verbatim}

\begin{Shaded}
\begin{Highlighting}[]
\CommentTok{\# Comprehensive tidy summary using skimr}
\FunctionTok{skim}\NormalTok{(starwars\_selected)}
\end{Highlighting}
\end{Shaded}

\begin{longtable}[]{@{}ll@{}}
\caption{Data summary}\tabularnewline
\toprule\noalign{}
\endfirsthead
\endhead
\bottomrule\noalign{}
\endlastfoot
Name & starwars\_selected \\
Number of rows & 87 \\
Number of columns & 5 \\
\_\_\_\_\_\_\_\_\_\_\_\_\_\_\_\_\_\_\_\_\_\_\_ & \\
Column type frequency: & \\
character & 2 \\
numeric & 3 \\
\_\_\_\_\_\_\_\_\_\_\_\_\_\_\_\_\_\_\_\_\_\_\_\_ & \\
Group variables & None \\
\end{longtable}

\textbf{Variable type: character}

\begin{longtable}[]{@{}
  >{\raggedright\arraybackslash}p{(\linewidth - 14\tabcolsep) * \real{0.1944}}
  >{\raggedleft\arraybackslash}p{(\linewidth - 14\tabcolsep) * \real{0.1389}}
  >{\raggedleft\arraybackslash}p{(\linewidth - 14\tabcolsep) * \real{0.1944}}
  >{\raggedleft\arraybackslash}p{(\linewidth - 14\tabcolsep) * \real{0.0556}}
  >{\raggedleft\arraybackslash}p{(\linewidth - 14\tabcolsep) * \real{0.0556}}
  >{\raggedleft\arraybackslash}p{(\linewidth - 14\tabcolsep) * \real{0.0833}}
  >{\raggedleft\arraybackslash}p{(\linewidth - 14\tabcolsep) * \real{0.1250}}
  >{\raggedleft\arraybackslash}p{(\linewidth - 14\tabcolsep) * \real{0.1528}}@{}}
\toprule\noalign{}
\begin{minipage}[b]{\linewidth}\raggedright
skim\_variable
\end{minipage} & \begin{minipage}[b]{\linewidth}\raggedleft
n\_missing
\end{minipage} & \begin{minipage}[b]{\linewidth}\raggedleft
complete\_rate
\end{minipage} & \begin{minipage}[b]{\linewidth}\raggedleft
min
\end{minipage} & \begin{minipage}[b]{\linewidth}\raggedleft
max
\end{minipage} & \begin{minipage}[b]{\linewidth}\raggedleft
empty
\end{minipage} & \begin{minipage}[b]{\linewidth}\raggedleft
n\_unique
\end{minipage} & \begin{minipage}[b]{\linewidth}\raggedleft
whitespace
\end{minipage} \\
\midrule\noalign{}
\endhead
\bottomrule\noalign{}
\endlastfoot
gender & 4 & 0.95 & 8 & 9 & 0 & 2 & 0 \\
species & 4 & 0.95 & 3 & 14 & 0 & 37 & 0 \\
\end{longtable}

\textbf{Variable type: numeric}

\begin{longtable}[]{@{}
  >{\raggedright\arraybackslash}p{(\linewidth - 20\tabcolsep) * \real{0.1707}}
  >{\raggedleft\arraybackslash}p{(\linewidth - 20\tabcolsep) * \real{0.1220}}
  >{\raggedleft\arraybackslash}p{(\linewidth - 20\tabcolsep) * \real{0.1707}}
  >{\raggedleft\arraybackslash}p{(\linewidth - 20\tabcolsep) * \real{0.0854}}
  >{\raggedleft\arraybackslash}p{(\linewidth - 20\tabcolsep) * \real{0.0854}}
  >{\raggedleft\arraybackslash}p{(\linewidth - 20\tabcolsep) * \real{0.0366}}
  >{\raggedleft\arraybackslash}p{(\linewidth - 20\tabcolsep) * \real{0.0732}}
  >{\raggedleft\arraybackslash}p{(\linewidth - 20\tabcolsep) * \real{0.0488}}
  >{\raggedleft\arraybackslash}p{(\linewidth - 20\tabcolsep) * \real{0.0732}}
  >{\raggedleft\arraybackslash}p{(\linewidth - 20\tabcolsep) * \real{0.0610}}
  >{\raggedright\arraybackslash}p{(\linewidth - 20\tabcolsep) * \real{0.0732}}@{}}
\toprule\noalign{}
\begin{minipage}[b]{\linewidth}\raggedright
skim\_variable
\end{minipage} & \begin{minipage}[b]{\linewidth}\raggedleft
n\_missing
\end{minipage} & \begin{minipage}[b]{\linewidth}\raggedleft
complete\_rate
\end{minipage} & \begin{minipage}[b]{\linewidth}\raggedleft
mean
\end{minipage} & \begin{minipage}[b]{\linewidth}\raggedleft
sd
\end{minipage} & \begin{minipage}[b]{\linewidth}\raggedleft
p0
\end{minipage} & \begin{minipage}[b]{\linewidth}\raggedleft
p25
\end{minipage} & \begin{minipage}[b]{\linewidth}\raggedleft
p50
\end{minipage} & \begin{minipage}[b]{\linewidth}\raggedleft
p75
\end{minipage} & \begin{minipage}[b]{\linewidth}\raggedleft
p100
\end{minipage} & \begin{minipage}[b]{\linewidth}\raggedright
hist
\end{minipage} \\
\midrule\noalign{}
\endhead
\bottomrule\noalign{}
\endlastfoot
height & 6 & 0.93 & 174.60 & 34.77 & 66 & 167.0 & 180 & 191.0 & 264 &
▂▁▇▅▁ \\
mass & 28 & 0.68 & 97.31 & 169.46 & 15 & 55.6 & 79 & 84.5 & 1358 &
▇▁▁▁▁ \\
birth\_year & 44 & 0.49 & 87.57 & 154.69 & 8 & 35.0 & 52 & 72.0 & 896 &
▇▁▁▁▁ \\
\end{longtable}

Let's look at a visual pattern of the height of different characters in
Star Wars.

\begin{Shaded}
\begin{Highlighting}[]
\NormalTok{starwars }\SpecialCharTok{\%\textgreater{}\%}
  \FunctionTok{ggplot}\NormalTok{(}\FunctionTok{aes}\NormalTok{(}\AttributeTok{x =}\NormalTok{ height)) }\SpecialCharTok{+}
  \FunctionTok{geom\_histogram}\NormalTok{(}\AttributeTok{binwidth =} \DecValTok{8}\NormalTok{, }\AttributeTok{fill =} \StringTok{"navy"}\NormalTok{, }\AttributeTok{color =} \StringTok{"blue"}\NormalTok{) }\SpecialCharTok{+}
  \FunctionTok{labs}\NormalTok{(}
    \AttributeTok{title =} \StringTok{"Figure 4. Histogram of Height of Characters in Star Wars"}\NormalTok{,}
    \AttributeTok{x =} \StringTok{"Height (cm)"}\NormalTok{,}
    \AttributeTok{y =} \StringTok{"Count"}
\NormalTok{  ) }\SpecialCharTok{+}
  \FunctionTok{theme\_minimal}\NormalTok{()}
\end{Highlighting}
\end{Shaded}

\begin{verbatim}
Warning: Removed 6 rows containing non-finite outside the scale range
(`stat_bin()`).
\end{verbatim}

\pandocbounded{\includegraphics[keepaspectratio]{handouttry_files/figure-pdf/unnamed-chunk-10-1.pdf}}

Now, we filter the species to get visual pattern of the height of
different \emph{human} characters in Star Wars.

\begin{Shaded}
\begin{Highlighting}[]
\NormalTok{starwars }\SpecialCharTok{\%\textgreater{}\%}
  \FunctionTok{filter}\NormalTok{(species }\SpecialCharTok{==} \StringTok{"Human"}\NormalTok{) }\SpecialCharTok{\%\textgreater{}\%}
  \FunctionTok{ggplot}\NormalTok{(}\FunctionTok{aes}\NormalTok{(}\AttributeTok{x =}\NormalTok{ height)) }\SpecialCharTok{+}
  \FunctionTok{geom\_histogram}\NormalTok{(}\AttributeTok{binwidth =} \DecValTok{8}\NormalTok{, }\AttributeTok{fill =} \StringTok{"navy"}\NormalTok{, }\AttributeTok{color =} \StringTok{"blue"}\NormalTok{) }\SpecialCharTok{+}
  \FunctionTok{labs}\NormalTok{(}
    \AttributeTok{title =} \StringTok{"Figure 5. Histogram of Height of Human Characters in Star Wars"}\NormalTok{,}
    \AttributeTok{x =} \StringTok{"Height (cm)"}\NormalTok{,}
    \AttributeTok{y =} \StringTok{"Count"}
\NormalTok{  ) }\SpecialCharTok{+}
  \FunctionTok{theme\_minimal}\NormalTok{()}
\end{Highlighting}
\end{Shaded}

\begin{verbatim}
Warning: Removed 5 rows containing non-finite outside the scale range
(`stat_bin()`).
\end{verbatim}

\pandocbounded{\includegraphics[keepaspectratio]{handouttry_files/figure-pdf/unnamed-chunk-11-1.pdf}}

Furthermore, this YouTube video
\href{https://www.youtube.com/watch?v=4vSfbz9YMa0}{Return of the Star
Wars dataset} may be an interesting resource to help you better
understand the dataset.

\section{Limitation}\label{limitation}

Summary statistics provide a quick and accessible overview of a dataset,
but they come with important limitations.

As highlighted in Naked Statistics
(\citeproc{ref-wheelan2013naked}{Wheelan, 2013}), these measures can be
misapplied, misinterpreted, or even manipulated, leading to misleading
conclusions. Summary statistics only describe what is present in the
data---they do not explain underlying causes, cannot be generalized
beyond the sample without further analysis, and offer no predictive
power. By condensing complex data into single values, they may obscure
important patterns or differences within the data. Additionally, summary
statistics do not reveal relationships between variables and can mask
issues like bias or subgroup variation.

Therefore, while useful for initial exploration, summary statistics
should always be complemented by more detailed analyses and
visualizations to avoid oversimplification and misinterpretation
(\citeproc{ref-wienclaw2009misuse}{Wienclaw, 2009}).

\section{Conclusion}\label{conclusion}

Summary statistics are a vital first step in data analysis, offering a
fast and accessible way to understand and interpret datasets. In R,
calculating measures like the mean, median, and standard deviation is
straightforward, whether for an entire dataset or for specific groups,
thanks to built-in functions and powerful packages such as dplyr.
Automating these summaries in R, as noted by
(\citeproc{ref-lane2013descriptive}{Lane, 2013}), streamlines your
workflow and helps organize your analysis.

However, it is important to recognize the limitations of summary
statistics. While they provide useful snapshots, they do not explain
underlying causes, predict future outcomes, or reveal relationships
between variables. Summary statistics can also obscure important
patterns or differences within subgroups and may mask issues like bias
or sampling problems, as highlighted in Naked Statistics
(\citeproc{ref-wheelan2013naked}{Wheelan, 2013}). Relying solely on
these measures can therefore lead to oversimplification or
misinterpretation of your data
(\citeproc{ref-wienclaw2009misuse}{Wienclaw, 2009}).

For these reasons, summary statistics should be viewed as an essential
starting point, but always complemented with more detailed analyses and
visualizations to gain a deeper, more accurate understanding of your
data. By mastering summary statistics in R---and remaining aware of
their limitations---you can uncover valuable insights, make more
informed decisions, and communicate your findings clearly in research or
business contexts.

\newpage

\section{References}\label{references}

\phantomsection\label{refs}
\begin{CSLReferences}{1}{0}
\bibitem[\citeproctext]{ref-lane2013descriptive}
Lane, D. M. (2013). Descriptive statistics. In \emph{Introduction to
statistics}. Rice University.
\url{https://onlinestatbook.com/2/introduction/descriptive.html}

\bibitem[\citeproctext]{ref-medcalf2018favourite}
Medcalf, A. (2018). \emph{My favourite r package for: Summarising data}.
\url{https://dabblingwithdata.amedcalf.com/2018/01/02/my-favourite-r-package-for-summarising-data/}

\bibitem[\citeproctext]{ref-oh2023making}
Oh, D. M., \& Pyrczak, F. (2023). \emph{Making sense of statistics: A
conceptual overview}. Routledge.

\bibitem[\citeproctext]{ref-Schork2021}
Schork, J. (2021). \emph{How to calculate summary statistics for the
columns of a data frame in r (example code)}. YouTube; Statistics Globe.
\url{https://www.youtube.com/watch?v=FMRkUqy1Sjw}

\bibitem[\citeproctext]{ref-dre2024gentle}
Videos, D. E. R. (2024). \emph{Gentle r \#4: Basic summary statistics in
r with r studio {[}video{]}}. YouTube.
\url{https://www.youtube.com/watch?v=8XFmPP93w_Y}

\bibitem[\citeproctext]{ref-walker2023gtsummary}
Walker, L. (2023). \emph{Easy summary tables in r with gtsummary
{[}video{]}}. YouTube. \url{https://www.youtube.com/watch?v=gohF7pp2XCg}

\bibitem[\citeproctext]{ref-wheelan2013naked}
Wheelan, C. (2013). \emph{Naked statistics: Stripping the dread from the
data}. W. W. Norton \& Company.

\bibitem[\citeproctext]{ref-dplyr2023}
Wickham, H., François, R., Henry, L., Müller, K., \& Vaughan, D. (2023).
\emph{Dplyr: A grammar of data manipulation}.
\url{https://dplyr.tidyverse.org/articles/dplyr.html}

\bibitem[\citeproctext]{ref-wienclaw2009misuse}
Wienclaw, R. A. (2009). The misuse of statistics. \emph{The Research
Starters Sociology}, 1--5.

\end{CSLReferences}

\newpage

\section{Affidative}\label{affidative}

I hereby affirm that this submitted paper was authored unaided and
solely by me. Additionally, no other sources than those in the reference
list were used. Parts of this paper, including tables and figures, that
have been taken either verbatim or analogously from other works have in
each case been properly cited with regard to their origin and
authorship. This paper either in parts or in its entirety, be it in the
same or similar form, has not been submitted to any other examination
board and has not been published.

I acknowledge that the university may use plagiarism detection software
to check my thesis. I agree to cooperate with any investigation of
suspected plagiarism and to provide any additional information or
evidence requested by the university.

Checklist:

\begin{itemize}
\tightlist
\item[$\boxtimes$]
  The handout contains 3-5 pages of text.
\item[$\boxtimes$]
  The submission contains the Quarto file of the handout.
\item[$\boxtimes$]
  The submission contains the Quarto file of the presentation.
\item[$\boxtimes$]
  The submission contains the HTML file of the handout.
\item[$\boxtimes$]
  The submission contains the HTML file of the presentation.
\item[$\boxtimes$]
  The submission contains the PDF file of the handout.
\item[$\boxtimes$]
  The submission contains the PDF file of the presentation.
\item[$\boxtimes$]
  The title page of the presentation and the handout contain personal
  details (name, email, matriculation number).
\item[$\boxtimes$]
  The handout contains a abstract.
\item[$\boxtimes$]
  The presentation and the handout contain a bibliography, created using
  BibTeX with APA citation style.
\item[$\boxtimes$]
  Either the handout or the presentation contains R code that proof the
  expertise in coding.
\item[$\boxtimes$]
  The handout includes an introduction to guide the reader and a
  conclusion summarizing the work and discussing potential further
  investigations and readings, respectively.
\item[$\boxtimes$]
  All significant resources used in the report and R code development.
\item[$\boxtimes$]
  The filled out Affidavit.
\item[$\boxtimes$]
  A concise description of the successful use of Git and GitHub, as
  detailed here: \url{https://github.com/hubchev/make_a_pull_request}.
\item[$\boxtimes$]
  The link to the presentation and the handout published on GitHub.
\end{itemize}

Jelin George, 2025,May 28, Cologne






\end{document}
